\documentclass{article}
\renewcommand{\subparagraph}{\paragraph}
\usepackage{include/nips13submit_e}
%\usepackage{theapa}
\usepackage{times}
\usepackage{amsmath, amsthm, amssymb}
\usepackage{pstricks}
\usepackage{pst-tree}
%\usepackage{color}
%\usepackage{makeidx}  % allows for indexgeneration
\usepackage{bm}
\usepackage{graphicx}
\usepackage{epsfig}
\usepackage{subfigure}
\usepackage{amsmath}
\usepackage{amsfonts}
\usepackage{color}
\usepackage{array}
\usepackage{colortbl}
\usepackage{framed}
\usepackage{url}
\usepackage{booktabs}
\usepackage{multirow}
%\usepackage{sgame}
\usepackage{dsfont}
\def\sgtextcolor{black}
\def\sglinecolor{black}
%\renewcommand{\gamestretch}{2}
\usepackage{multicol}
\usepackage{lscape}
\usepackage{relsize}
\usepackage{rotating}
\usepackage{tikz}
\usetikzlibrary{calc}

% ============== Mike's commands ==============
\usepackage{nicefrac}
\newcommand{\vect}[1]{\underline{\smash{#1}}}
\renewcommand{\v}[1]{\vect{#1}}
\newcommand{\reals}{\mathds{R}}
\newcommand{\sX}{\mathcal{X}}
\newcommand{\sD}{\mathcal{D}}
\newcommand{\br}{^{\text{\textnormal{ r}}}}
\newcommand{\cat}{^{\text{\textnormal{c}}}}
% ============== ==============

\newcommand{\cut}[1]{}
\newcommand{\hide}[1]{}
\renewcommand{\blue}[1]{{\textcolor{blue}{#1}}}
%\renewcommand{\blue}[1]{#1}

%% Zapf Chancery font: has lowercase script letters
\DeclareFontFamily{OT1}{pzc}{}
\DeclareFontShape{OT1}{pzc}{m}{it}{<-> s * [1.200] pzcmi7t}{}
\DeclareMathAlphabet{\mathscr}{OT1}{pzc}{m}{it}

\newcommand\transpose{{\textrm{\tiny{\sf{T}}}}}
\newcommand{\note}[1]{}
\newcommand{\hlinespace}{~\vspace*{-0.15cm}~\\\hline\\\vspace*{0.15cm}}
%\newcommand{\hlinespace}{~\vspace*{0.45cm}\\\hline\\~\vspace*{-0.9cm}}
%\newcommand{\hlinespace}{~\vspace*{0.05cm}\\\hline~\vspace*{0.5cm}}

% comment the next line to turn off notes
\renewcommand{\note}[1]{~\\\frame{\begin{minipage}[c]{\textwidth}\vspace{2pt}\center{#1}\vspace{2pt}\end{minipage}}\vspace{3pt}\\}
\newcommand{\lnote}[1]{\note{#1}}
\newcommand{\emcite}[1]{\citet{#1}}
\newcommand{\yrcite}[1]{\citeyear{#1}}
\newcommand{\aunpcite}[1]{\citeR{#1}}

\newcommand{\heavyrule}{\specialrule{\heavyrulewidth}{.4em}{.4em}}
\newcommand{\lightrule}{\specialrule{.03em}{.4em}{.4em}}

%%%%%%%%%%%%%%%%%%%%%%%%%%%%%%%%%%%%%%%%%%%%%%

%% keep figures from going onto a page by themselves
\renewcommand{\topfraction}{0.9}
\renewcommand{\textfraction}{0.07}
\renewcommand{\floatpagefraction}{0.9}
\renewcommand{\dbltopfraction}{0.9}      % for double-column styles
\renewcommand{\dblfloatpagefraction}{0.7}   % for double-column styles



\usepackage{amsmath, amsthm, amssymb}
\newtheorem{thm}{Theorem}%[section]
\newtheorem{lem}[thm]{Lemma}
\newtheorem{prop}[thm]{Proposition}
\newtheorem{cor}[thm]{Corollary}
\newtheorem{obs}[thm]{Observation}

%\theoremstyle{definition}
\newtheorem{define}[thm]{Definition}
\hyphenation{ge-ne-ral-ize}




\newcommand{\Var}{\ensuremath\text{Var}}
\newcommand{\indicator}{\ensuremath\mathds{I}}

\newcommand{\fhspace}{\vspace*{0.2cm}}
\newcommand{\newsec}{\hspace{0cm}}

% replaces tabular; takes same arguments. use \midrule for a rule, no vertical rules, and eg \cmidrule(l){2-3} as needed with \multicolumn
\newenvironment{ktabular}[1]{\sffamily\small\begin{center}\begin{tabular}[c]{#1}\toprule}{\bottomrule \end{tabular}\end{center}\normalsize\rmfamily\vspace{-5pt}}
\newcommand{\tbold}[1]{\textbf{#1}}
\newcommand{\interrowspace}{.6em}


\begin{document}

\title{Raiders of the Lost Architecture:\\A Kernel for Conditional Parameter Spaces}

\author{Frank Hutter and Michael A. Osborne\\
{\tt fh@informatik.uni-freiburg.de} and {\tt mosb@robots.ox.ac.uk}
}

\maketitle
\begin{abstract}
When performing model-based optimization, we must often search over structures with differing numbers of parameters.  For instance, we may wish to search over neural network architectures with an unkown number of layers.  To combine information between different architectures, we define a family of kernels for conditional parameter spaces.
\end{abstract}



%%%%%%%%%%%%%%%%%%%%%%%%%%%%%%%%%%%%%%%%%%%%%%%%%%%%%%%%%%%%%%%%%%%%%%%%%%%%%%%%%%%%%%%%%%%%%%%%%
\section{Introduction}
%%%%%%%%%%%%%%%%%%%%%%%%%%%%%%%%%%%%%%%%%%%%%%%%%%%%%%%%%%%%%%%%%%%%%%%%%%%%%%%%%%%%%%%%%%%%%%%%%

\cite{rasmussen38gaussian}




\section{A Kernel for Conditional Parameter Spaces}
We aim to do inference about some function $g$ with domain (input space) $\sX$. $\sX = \prod_{i=1}^D \sX_i$ is a $D$-dimensional input space, where each individual dimension is either bounded real or categorical, that is, $\sX_i$ is either $[l_i, u_i] \subset \reals$ (with lower and upper bounds $l_i$ and $u_i$, respectively) or $\{v_{i,1}, \dots, v_{i,m_i}\}$. 

Associated with $\sX$, there is a DAG structure $\sD$, whose vertices are the dimensions $\{1,\,\ldots,\,D\}$. $\sX$ will be restricted by $\sD$: if vertex $i$ has children under $\sD$, $\sX_i$ must be categorical. $\sD$ is also used to specify when each input is \emph{active} (that is, relevant to inference about $g$). In particular, we assume each input dimension is only active under some instantiations of its ancestor dimensions in $\sD$. More precisely, we define $D$ functions $\delta_i\colon \sX\to \mathcal{B}$, for $i \in \{1,\,\ldots,\,D\}$, and where $\mathcal{B} = \{\text{true}, \text{false}\}$. We take 
\begin{equation}
 \delta_i(\v{x}) = \delta_i\bigl(\v{x}(\text{anc}_i)\bigr),
\end{equation}
where $\text{anc}_i$ are the ancestor vertices of $i$ in $\sD$, such that $\delta_i(\v{x})$ is true only for appropriate values of those entries of $\v{x}$ corresponding to ancestors of $i$ in $\sD$. We say $i$ is active for $\v{x}$ iff $\delta_i(\v{x})$.

%if all its parent dimensions $P_i$ are active themselves and each parent $p\in P_i$ takes one of the values in the finite set $V_{i,p}$. 
Our aim is to specify a kernel for $\sX$, \emph{i.e.}, a positive semi-definite function  $k\colon \sX \times \sX \to \reals$. We will first specify an individual kernel for each input dimension, \emph{i.e.}, a positive semi-definite function $k_i\colon \sX \times \sX \to \reals$. $k$ can then be taken as either a sum,
\begin{equation}
 k(\v{x}, \v{x}') = \sum_{i=1}^D k_i(\v{x},\v{x}'),
\end{equation}
product,
\begin{equation}
 k(\v{x}, \v{x}') = \prod_{i=1}^D k_i(\v{x},\v{x}'),
\end{equation}
or any other permitted combination, of these individual kernels. Note that each individual kernel $k_i$ will depend on an input vector $\v{x}$ only through dependence on $x_i$ and $\delta_i(\v{x})$,
\begin{equation}
  k_i(\v{x},\v{x}') = \tilde{k}_i\bigl(x_i,\delta_i(\v{x}),x_i', \delta_i(\v{x}') \bigr).
\end{equation}
That is, $x_j$ for $j\neq i$ will influence $k_i(\v{x},\v{x}')$ only if $j \in \text{anc}_i$, and only by affecting whether $i$ is active.

Below we will construct pseudometrics $d{_i}\colon \sX \times \sX \to \reals^+$: that is, $d_i$ satisfies the requirements of a metric aside from the identity of indiscernibles. As for $k_i$, these pseudometrics will depend on an input vector $\v{x}$ only through dependence on both $x_i$ and $\delta_i(\v{x})$. $d{_i}(\v{x}, \v{x}')$ will be designed to provide an intuitive measure of how different $g(\v{x})$ is from $g(\v{x}')$. 
For each $i$, we will then construct a (pseudo-)isometry $f_i$ from
$\sX$ 
to a Euclidean space ($\reals^2$ for bounded real parameters, and $\reals^m$ for categorical-valued parameters with $m$ choices). That is, denoting the Euclidean metric on the appropriate space as $d{_E}$, $f_i$ will be such that
\begin{equation}
\label{eqn:d_i}
 d{_i}(\v{x},\v{x}')
=
d_{\text{E}}(f{_i}\bigl(\v{x}), f{_i}(\v{x}')\bigr)
\end{equation}
for all $\v{x}, \v{x}' \in \sX$. We can then use our transformed inputs, $f_i(\v{x})$, within any standard Euclidean kernel $\kappa$. We'll make this explicit in Proposition \ref{prop:psd_if_isometry}. 




We'll now define pseudometrics $d_i$ and associated isometries $f_i$ for both the bounded real and categorical cases. 


%%%%%%%%%%%%%%%%%%%%%%%%%%%%%%%%%%%%%%%%%%%%%%%%%%%%%%%%%%%%%%%%%%%%%%%%%%%%%%%%%%%%%%%%%%%%%%%%%
\paragraph{Bounded Real Dimensions}
%%%%%%%%%%%%%%%%%%%%%%%%%%%%%%%%%%%%%%%%%%%%%%%%%%%%%%%%%%%%%%%%%%%%%%%%%%%%%%%%%%%%%%%%%%%%%%%%%



\begin{figure}
% A simple figure to illustrate Mike and Frank's embedding
% Sept 2013

\newcommand{\scaleamount}{2.5}
\newcommand{\anglemin}{40}
\newcommand{\angleminplusone}{41}
\newcommand{\anglemax}{150}
\newcommand{\angleone}{70}
\newcommand{\angletwo}{120}
\newcommand{\bigradius}{3}
\newcommand{\biggerradius}{3.3}
\newcommand{\smallradius}{0.5}
\newcommand{\xtwolabelangle}{90}
\newcommand{\belowamount}{0.2cm}
\newcommand{\belowamounttwo}{0.11cm}
\newcommand{\myell}{l}
\newcommand{\embedding}{g}

%\begin{center}
%\framebox{
\begin{tikzpicture}
[trans/.style={thick,->,shorten >=2pt,shorten <=2pt,>=stealth},scale=\scaleamount]

\tikzset{state/.style={circle,draw=black, very thick,minimum size=4em}}

\def\centerarc[#1] (#2) (#3:#4:#5)% [draw options] (center) (initial angle:final angle:radius)
{ \draw[#1] (#2) ++(#3:#5) arc (#3:#4:#5);
}
	\coordinate (left) at ({\bigradius*cos(\anglemax)}, {\bigradius*sin(\anglemax)});
	\coordinate (right) at ({\bigradius*cos(\anglemin)}, {\bigradius*sin(\anglemin)});

	\coordinate (O1) at (-2, 0);
	\coordinate (O2) at (3, 1.25);
	
	\coordinate (left1) at ($ (left) + (O1) $);
	\coordinate (right1) at ($ (right) + (O1) $);
	\coordinate (left2) at ($ (left) + (O2) $);
	\coordinate (right2) at ($ (right) + (O2) $);
	
	% First arc
	% ==================
	
	% Draw arcs
	\centerarc[thick] (O1) (\anglemin:\anglemax:\bigradius)
	\centerarc[thick] (O1) (\anglemin:\anglemax:\smallradius)

	\draw[fill] (left1) circle (1.5pt);
	\draw (left1) node[below = \belowamount, right] {$\embedding(0, \textnormal{true}, u$)};
	
	\draw[fill] (right1) circle (1.5pt);
	\draw (right1) node[below = \belowamount] (r1below) {}
	               node[left] {$\embedding(0, \textnormal{true}, \myell)$};

	\draw[fill] (O1) circle (1.5pt);
	\draw (O1) node[below = \belowamount] (o1below) {}
	           node[left = 0.3cm] {$\embedding(0, \textnormal{false}, \cdot)$};

	\draw (left1) -- (O1) node[above] {$\rho \pi$};
	\draw (right1) -- (O1) node[midway, below] {$\omega$};
	
	
	% Second arc
	% ================================
	\centerarc[thick] (O2) (\anglemin:\anglemax:\bigradius)
	\centerarc[thick] (O2) (\anglemin:\anglemax:\smallradius)

	\draw[fill] (left2) circle (1.5pt);
	
	\draw[fill] (right2) circle (1.5pt);
	\draw (right2) node[below = \belowamount] (r2below) {};
	

	\draw[fill] (O2) circle (1.5pt);
	\draw (O2) node[below = \belowamount] (o2below) {}
                   node[right] {$\embedding(1, \textnormal{false}, \cdot)$};

	\draw[dashed] (left2) -- (O2);
	\draw[dashed] (right2) -- (O2);


	% Connect the arcs
	\draw (left1) -- (left2);
	\draw (right1) -- (right2);
	\draw[thick] (O1) -- (O2);
	
	% Draw surface
	\foreach \i in {\angleminplusone, ..., \anglemax}
	{
		\coordinate (arca) at ({\bigradius*cos(\i)}, {\bigradius*sin(\i)});
		\coordinate (arcb) at ({\bigradius*cos(\i - 1)}, {\bigradius*sin(\i - 1)});
		\coordinate (arca1) at ($ (arca) + (O1) $);
		\coordinate (arca2) at ($ (arca) + (O2) $);
		\coordinate (arcb1) at ($ (arcb) + (O1) $);
		\coordinate (arcb2) at ($ (arcb) + (O2) $);
%		\draw[green] (arca1) -- (arca2);
		\fill[fill=blue!40,fill opacity=0.8](arca1)--(arca2)--(arcb2)--(arcb1)--cycle;
	}

	\draw (right2) node[left] {$\embedding(1, \textnormal{true}, \myell)$};

	% Draw arrows showing in which direction x1 varies.
	\coordinate(x1half) at ($ (right1)!0.5!(right2) $);
	\draw (x1half) node[below = \belowamounttwo] (x1halfbelow) {$x_1$};
	\draw[trans] ($ (r1below)!0.45!(r2below) $) -- ($ (r1below)!0.3!(r2below) $);
	\draw[trans] ($ (r1below)!0.55!(r2below) $) -- ($ (r1below)!0.7!(r2below) $);

	\coordinate(x2half) at ($ (O1)!0.6!(O2) $);
	\draw (x2half) node[below = \belowamounttwo] (x2halfbelow) {$x_1$};
	\draw[trans] ($ (o1below)!0.55!(o2below) $) -- ($ (o1below)!0.4!(o2below) $);
	\draw[trans] ($ (o1below)!0.65!(o2below) $) -- ($ (o1below)!0.8!(o2below) $);

	% Draw arrows showing in which direction x2 varies.
	\coordinate (arclabel) at ($ (O1) + ({\biggerradius*cos(\xtwolabelangle)}, {\biggerradius*sin(\xtwolabelangle)}) $);
	\draw (arclabel) node {$x_2$};
	\centerarc[trans] (O1) ( \xtwolabelangle + 5  :  \xtwolabelangle + 30  :\biggerradius)
	\centerarc[trans] (O1) ( \xtwolabelangle - 5  :  \xtwolabelangle - 30  :\biggerradius)

%	\coordinate (x1) at ({\bigradius*cos(\angletwo)}, {\bigradius*sin(\angletwo)});
%	\coordinate (x2) at ({\bigradius*cos(\angleone)}, {\bigradius*sin(\angleone)});
%	\draw[fill] (x1) circle (1.5pt);
%	\draw (x1) node[above, left] {$f(x_1, \textnormal{true})$};
%	\draw[fill] (x2) circle (1.5pt);
%	\draw (x2) node[above, right] {$f(x_2, \textnormal{true})$};
\end{tikzpicture}
%}
%\end{center}

\caption{A demonstration of the embedding giving rise to the pseduo-metric: All points for which $\delta_i(x) =$ false are mapped onto a line depending only on $x_1$.  Points for which $\delta_i(x) =$ true are mapped to the surfrace of a semicylinder, having an extra dimension $x_2$.  This embedding gives a constant distance between pairs of points which have differing values of $\delta$ but the same values of $x_1$.  The parameter $\rho$ determines how much distance there is along the arc.}
\end{figure}



Let's first focus on a bounded real input dimension $i$, i.e., $\sX_i=[l_i, u_i]$.
To emphasize that we're in this real case, we explicitly denote the pseudometric as $d\br_i$ and the (pseudo-)isometry from $(\sX, d_i)$ to $\reals^2,d_\text{E}$ 
as $f\br_i$. For the definitions, recall that $\delta_i(\v{x})$ is true iff dimension $i$ is active given the instantiation of $i$'s ancestors in $\v{x}$.

\begin{eqnarray}
\nonumber{}f_i\br(\v{x}) & = & \left\{\begin{array}{ll}
[0,0]^\transpose & \textrm{ if } \delta_i(\v{x}) = \textrm{ false }\\
\nonumber{} \omega_i [\sin{\pi\rho_i\frac{x_i}{u_i-l_i}}, \cos{\pi\rho_i\frac{x_i}{u_i-l_i}}]^\transpose & \textrm{ otherwise.}\end{array}\right..
\end{eqnarray}

\begin{eqnarray}
\nonumber{}d\br_i(\v{x}, \v{x}') & = & \left\{\begin{array}{ll}
\nonumber{} 0 & \textrm{ if } \delta_i(\v{x}) = \delta_i(\v{x}') = \textrm{false}\\
\nonumber{} \omega_i & \textrm{ if } \delta_i(\v{x}) \neq \delta_i(\v{x}')\\
\nonumber{} \omega_i \sqrt{2} \sqrt{1 - \cos(\pi\rho_i \frac{x_i-x_i'}{u_i-l_i})} & \textrm{ if } \delta_i(\v{x}) = \delta_i(\v{x}') = \textrm{true}. \end{array}\right.
\end{eqnarray}



Although our formal arguments do not rely on this, Proposition \ref{prop:dbr_pseudometric} in the appendix shows that $d\br_i$ is a pseudometric. 
This pseudometric is defined by two parameters: $\omega_i \in [0,1]$ and $\rho_i \in [0,1]$. We firstly define 
\begin{equation}\label{eq:gamma}
\omega_i = \prod_{j \in \text{anc}_i \cup \{i\}} \gamma_j, 
\end{equation}
where $\gamma_j \in [0,1]$. This encodes the intuitive notion that differences on lower levels of the hierarchy count less than differences in their ancestors.

Also note that, as desired, if $i$ is inactive for both $\v{x}$ and $\v{x}'$, $d\br_i$ specifies that $g(\v{x})$ and $g(\v{x}')$ should not differ owing to differences between $x_i$ and $x_i'$. Secondly, if $i$ is active for both $\v{x}$ and $\v{x}'$, the difference between $g(\v{x})$ and $g(\v{x}')$ due to $x_i$ and $x_i'$ increases monotonically with increasing $\left|x_i-x_i'\right|$. Parameter $\rho_i$ controls whether differing in the activity of $i$ contributes more or less to the distance than differing in $x_i$ should $i$ be active. If $\rho = \nicefrac{1}{3}$, and if $i$ is inactive for exactly one of $\v{x}$ and $\v{x}'$, $g(\v{x})$ and $g(\v{x}')$ are as different as is possible due to dimension $i$; that is, $g(\v{x})$ and $g(\v{x}')$ are exactly as different in that case as if $x_i=l_i$ and $x_i'=u_i$. For $\rho>\nicefrac{1}{3}$, $i$ being active for both $\v{x}$ and $\v{x}'$ means that $g(\v{x})$ and $g(\v{x}')$ could potentially be more different than if
$i$ was active in only one of them. For $\rho<\nicefrac{1}{3}$, the converse is true.
%\footnote{Note that $\v{x}$ and $\v{x}'$ must differ in at least one ancestor dimension of $i$ in order for $\delta_i(\v{x}) \neq \delta_i(\v{x}')$ to hold, such that in the final kernel combining kernels $k_i$ due to each dimension $i$, differences in the activity of dimension $i$ are penalized both in kernel $k_i$ and in the distance for the kernel of the ancestor dimension causing the difference in $i$'s activity.}

We now show that $d\br_i$ and $f\br_i$ can be plugged into a positive semi-definite kernel over Euclidean space to define a valid kernel over space $\sX$.

\begin{prop}
Let $\kappa$ be a positive semi-definite covariance function over Euclidean space.
Then, $k_i\colon \sX \times \sX\to \reals^+$, defined by 
%\[k_i(\v{x},\v{x}') = \kappa( d_{\text{E}}(f_i(\v{x}), f_i(\v{x}')) )\]
\[k_i(\v{x},\v{x}') = \kappa\bigl( d\br_i(\v{x}, \v{x}') \bigr)\]
is a positive semi-definite covariance function over input space $\sX$. 
\label{prop:cont_psd}
\end{prop}
A proof of proposition \ref{prop:cont_psd} can be found in the supplementary material.


\paragraph{Categorical Dimensions}  The pseduometric can also be extended to handle categorical dimensions - see the supplementary material.




\section{Experiments}

All the separate models split the data into 6 different datasets, one for each level, and build a separate model for each level.  The gp-hierarchical model gets all the data together because it can handle it.  

However, The gp-hierarchical model changes two things at once compared to the separate-gp-ard model: besides embedding the missing data in a different spot, it also has embeds the fully-observed data on semi-circles, and has a different parameterization.  
So, it could be the case that even when the data are fully observed, embedding the data on a semi-circle and using a different parameterization might cause better or worse performance than a standard squared-exp.  To find out if this is the case, we compare separate-gp-ard and separate-hierarchical to find out if these two models have different performance even in the standard fully-observed case.

% --- Automatically generated by resultsToLatex4.m ---
% Exported at 23-Oct-2013 14:29:55
\begin{center}
\begin{tabular}{l | r r r r}
Method & \rotatebox{0}{ NN   }  & \rotatebox{0}{ NN   log }  & \rotatebox{0}{ NN   half }  & \rotatebox{0}{ NN   log half }  \\ \hline
Separate Linear & $0.898$ & $0.874$ & $0.982$ & $\mathbf{3.482}$ \\
Separate {\sc gp} & $\mathbf{0.815}$ & $0.547$ & $1.012$ & $0.753$ \\
Poor Man's embedding Linear & $0.882$ & $0.661$ & $0.980$ & $\mathbf{0.732}$ \\
Poor Man's embedding {\sc gp} & $\mathbf{0.787}$ & $\mathbf{0.498}$ & $0.819$ & $\mathbf{0.656}$ \\
Separate Hierarchical {\sc gp} & $\mathbf{0.771}$ & $0.554$ & $\mathbf{0.983}$ & $0.847$ \\
Hierarchical {\sc gp} & $\mathbf{0.744}$ & $\mathbf{0.450}$ & $\mathbf{0.674}$ & $\mathbf{0.607}$ \\
\end{tabular}
\end{center}
% End automatically generated LaTeX



\section{Conclusion}



\bibliography{hierarchicalkernel}
\bibliographystyle{unsrt}


\end{document}

